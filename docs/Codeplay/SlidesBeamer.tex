% This presentation is based on a Beamer theme from Seth Brown, distributed
% under the following license:
%
% ----------------------------------------------------------------------------
% This program can be redistributed and/or modified under the terms
% of the GNU Public License, version 3.
%
% Seth Brown, Ph.D.
% sethbrown@drbunsen.org

\documentclass[t]{beamer}

\usepackage{graphicx} % graphics
\usepackage{hyperref} % urls
\usepackage{booktabs, caption} % table styling
\usepackage[utf8x]{inputenc}
\usepackage{fancyvrb}
\usepackage{tcolorbox}

% suppress navigation bar
\beamertemplatenavigationsymbolsempty

\mode<presentation>
{
  \usetheme{codeplay}
  \setbeamercovered{transparent}
  \setbeamertemplate{items}[circle]
}

% set fonts
\usepackage{fontspec}
\setsansfont{Calibri}
\setmonofont{Consolas}
\setbeamerfont{frametitle}{size=\LARGE,series=\bfseries}

% color definitions
\usepackage{color}
\definecolor{uiwhite}{RGB}{255, 255, 255}
\definecolor{uidarkgray}{RGB}{24,24,25}
\definecolor{uigray}{RGB}{51,51,51}
\definecolor{uilightgray}{RGB}{123,123,123}
\definecolor{uicyan}{RGB}{0,255,255}
\definecolor{uipink}{RGB}{255,153,255}

% caption styling
\DeclareCaptionFont{uiwhite}{\color{uiwhite}}
\captionsetup{labelfont={uiwhite},textfont=uiwhite}

% title slide definition
\title{Building an LLVM Backend}
\subtitle{LLVM 2014 tutorial}
\author{Fraser Cormak \\ Pierre-Andre Saulais}
\institute{Codeplay Software @codeplaysoft}

\date{\today}

% Code box formatting
\fvset{fontsize=\scriptsize,tabsize=4}
\tcbset{colback=uigray,colframe=uilightgray,colupper=uiwhite,left=1pt,right=1pt,top=1pt,bottom=1pt,arc=0pt,toprule=1pt,bottomrule=1pt,leftrule=1pt,rightrule=1pt}

\newenvironment{codebox}
 {\VerbatimEnvironment
  \begin{tcolorbox}%
  \begin{Verbatim}}
 {\end{Verbatim}\end{tcolorbox}}

% Code caption that comes after the code box.
\newcommand{\codecaption}[2][-4.4ex]{%
  \vspace{#1}
  \hfill\mbox{\footnotesize{\textcolor{uicyan}{#2}}\hspace{0.5em}}
  \vspace*{2ex}}

\newcommand{\sourcebox}[3][firstline=1]{%
  \begin{tcolorbox}\VerbatimInput[#1]{#2}\end{tcolorbox}
  \codecaption{#3}}

\newcommand{\examplebox}[2][firstline=1]{\sourcebox[#1]{examples/#2}{#2}}

\newcommand{\codeempha}[1]{\textcolor{uipink}{#1}}

% Create a slide for a part and include the source code.
\newcommand{\talkpart}[3]{%
\section{Part #1: #2}

\begin{frame}[c]{Part #1}

\centerline{\LARGE{#2}}

\end{frame}
\input{#3}}

%%%%%%%%%%%%%%%%%%%%%%%%%%%%%%%%%%%%%%%%%%%%%%%%%%%%%%%%%%%%%%%%%%%%%%%%%%%%%%%%

\begin{document}

\setbeamertemplate{background}
{\includegraphics[width=\paperwidth,height=\paperheight]{dark_background_title.png}}
\setbeamertemplate{footline}[default]

\begin{frame}
\vspace{2cm}
  \titlepage
  \vspace{10cm}
\end{frame}

%%%%%%%%%%%%%%%%%%%%%%%%%%%%%%%%%%%%%%%%%%%%%%%%%%%%%%%%%%%%%%%%%%%%%%%%%%%%%%%%

% Set the background for the rest of the slides.
%\setbeamertemplate{background}
% {\includegraphics[width=\paperwidth,height=\paperheight]{dark_background.png}}
\setbeamertemplate{background}{}
\setbeamertemplate{footline}[codeplaytheme]

\section{Introduction}

\begin{frame}{Introduction}

\begin{itemize}
    \item Yet another talk about creating a LLVM target?
    \item LLVM backend crash course, for beginners
    \begin{itemize}
        \item How-tos and tips
        \item Solution to common problems
    \end{itemize}  
    \item Example target created for this tutorial
    \begin{itemize}
        \item Can be used to see how LLVM works
        \item Can be used as a skeleton to bootstrap new target
    \end{itemize}
\end{itemize}

\end{frame}

%%%%%%%%%%%%%%%%%%%%%%%%%%%%%%%%%%%%%%%%%%%%%%%%%%%%%%%%%%%%%%%%%%%%%%%%%%%%%%%%

\begin{frame}{Overview}

\tableofcontents[subsectionstyle=hide]

\end{frame}

%%%%%%%%%%%%%%%%%%%%%%%%%%%%%%%%%%%%%%%%%%%%%%%%%%%%%%%%%%%%%%%%%%%%%%%%%%%%%%%%

\talkpart{1}{Background}{SlidesBeamer-part1.tex}

\talkpart{2}{Creating your own target}{SlidesBeamer-part2.tex}

\talkpart{3}{How-tos for specific tasks}{SlidesBeamer-part3.tex}

\talkpart{4}{Troubleshooting and resources}{SlidesBeamer-part4.tex}

%%%%%%%%%%%%%%%%%%%%%%%%%%%%%%%%%%%%%%%%%%%%%%%%%%%%%%%%%%%%%%%%%%%%%%%%%%%%%%%%

\section{Conclusion}

\begin{frame}{Summary}

\begin{itemize}
    \item Should be enough to create a very simple target!
    \item Many things were not covered in this talk:
    \begin{itemize}
        \item Using different types and legalization
        \item Scheduling
        \item Intrinsics
        \item ...
    \end{itemize}
    \item Introduced resources to go further
\end{itemize}

\end{frame}

%%%%%%%%%%%%%%%%%%%%%%%%%%%%%%%%%%%%%%%%%%%%%%%%%%%%%%%%%%%%%%%%%%%%%%%%%%%%%%%%

\begin{frame}{Thank you!}

\begin{itemize}
    \item Q\&A
    \item Happy to answer questions by email too:
    \begin{itemize}
        \item \url{fraser@codeplay.com}
        \item \url{pierre-andre@codeplay.com}
    \end{itemize}
    \item Check out our code from GitHub:
    \begin{itemize}
        \item \url{github.com/frasercrmck/llvm-leg}
    \end{itemize}
\end{itemize}

\end{frame}

%%%%%%%%%%%%%%%%%%%%%%%%%%%%%%%%%%%%%%%%%%%%%%%%%%%%%%%%%%%%%%%%%%%%%%%%%%%%%%%%

\end{document}
